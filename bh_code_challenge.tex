% Options for packages loaded elsewhere
\PassOptionsToPackage{unicode}{hyperref}
\PassOptionsToPackage{hyphens}{url}
%
\documentclass[
]{article}
\usepackage{lmodern}
\usepackage{amssymb,amsmath}
\usepackage{ifxetex,ifluatex}
\ifnum 0\ifxetex 1\fi\ifluatex 1\fi=0 % if pdftex
  \usepackage[T1]{fontenc}
  \usepackage[utf8]{inputenc}
  \usepackage{textcomp} % provide euro and other symbols
\else % if luatex or xetex
  \usepackage{unicode-math}
  \defaultfontfeatures{Scale=MatchLowercase}
  \defaultfontfeatures[\rmfamily]{Ligatures=TeX,Scale=1}
\fi
% Use upquote if available, for straight quotes in verbatim environments
\IfFileExists{upquote.sty}{\usepackage{upquote}}{}
\IfFileExists{microtype.sty}{% use microtype if available
  \usepackage[]{microtype}
  \UseMicrotypeSet[protrusion]{basicmath} % disable protrusion for tt fonts
}{}
\makeatletter
\@ifundefined{KOMAClassName}{% if non-KOMA class
  \IfFileExists{parskip.sty}{%
    \usepackage{parskip}
  }{% else
    \setlength{\parindent}{0pt}
    \setlength{\parskip}{6pt plus 2pt minus 1pt}}
}{% if KOMA class
  \KOMAoptions{parskip=half}}
\makeatother
\usepackage{xcolor}
\IfFileExists{xurl.sty}{\usepackage{xurl}}{} % add URL line breaks if available
\IfFileExists{bookmark.sty}{\usepackage{bookmark}}{\usepackage{hyperref}}
\hypersetup{
  pdftitle={BH Code Challenge},
  pdfauthor={Allan Trapp II},
  hidelinks,
  pdfcreator={LaTeX via pandoc}}
\urlstyle{same} % disable monospaced font for URLs
\usepackage[margin=1in]{geometry}
\usepackage{color}
\usepackage{fancyvrb}
\newcommand{\VerbBar}{|}
\newcommand{\VERB}{\Verb[commandchars=\\\{\}]}
\DefineVerbatimEnvironment{Highlighting}{Verbatim}{commandchars=\\\{\}}
% Add ',fontsize=\small' for more characters per line
\usepackage{framed}
\definecolor{shadecolor}{RGB}{248,248,248}
\newenvironment{Shaded}{\begin{snugshade}}{\end{snugshade}}
\newcommand{\AlertTok}[1]{\textcolor[rgb]{0.94,0.16,0.16}{#1}}
\newcommand{\AnnotationTok}[1]{\textcolor[rgb]{0.56,0.35,0.01}{\textbf{\textit{#1}}}}
\newcommand{\AttributeTok}[1]{\textcolor[rgb]{0.77,0.63,0.00}{#1}}
\newcommand{\BaseNTok}[1]{\textcolor[rgb]{0.00,0.00,0.81}{#1}}
\newcommand{\BuiltInTok}[1]{#1}
\newcommand{\CharTok}[1]{\textcolor[rgb]{0.31,0.60,0.02}{#1}}
\newcommand{\CommentTok}[1]{\textcolor[rgb]{0.56,0.35,0.01}{\textit{#1}}}
\newcommand{\CommentVarTok}[1]{\textcolor[rgb]{0.56,0.35,0.01}{\textbf{\textit{#1}}}}
\newcommand{\ConstantTok}[1]{\textcolor[rgb]{0.00,0.00,0.00}{#1}}
\newcommand{\ControlFlowTok}[1]{\textcolor[rgb]{0.13,0.29,0.53}{\textbf{#1}}}
\newcommand{\DataTypeTok}[1]{\textcolor[rgb]{0.13,0.29,0.53}{#1}}
\newcommand{\DecValTok}[1]{\textcolor[rgb]{0.00,0.00,0.81}{#1}}
\newcommand{\DocumentationTok}[1]{\textcolor[rgb]{0.56,0.35,0.01}{\textbf{\textit{#1}}}}
\newcommand{\ErrorTok}[1]{\textcolor[rgb]{0.64,0.00,0.00}{\textbf{#1}}}
\newcommand{\ExtensionTok}[1]{#1}
\newcommand{\FloatTok}[1]{\textcolor[rgb]{0.00,0.00,0.81}{#1}}
\newcommand{\FunctionTok}[1]{\textcolor[rgb]{0.00,0.00,0.00}{#1}}
\newcommand{\ImportTok}[1]{#1}
\newcommand{\InformationTok}[1]{\textcolor[rgb]{0.56,0.35,0.01}{\textbf{\textit{#1}}}}
\newcommand{\KeywordTok}[1]{\textcolor[rgb]{0.13,0.29,0.53}{\textbf{#1}}}
\newcommand{\NormalTok}[1]{#1}
\newcommand{\OperatorTok}[1]{\textcolor[rgb]{0.81,0.36,0.00}{\textbf{#1}}}
\newcommand{\OtherTok}[1]{\textcolor[rgb]{0.56,0.35,0.01}{#1}}
\newcommand{\PreprocessorTok}[1]{\textcolor[rgb]{0.56,0.35,0.01}{\textit{#1}}}
\newcommand{\RegionMarkerTok}[1]{#1}
\newcommand{\SpecialCharTok}[1]{\textcolor[rgb]{0.00,0.00,0.00}{#1}}
\newcommand{\SpecialStringTok}[1]{\textcolor[rgb]{0.31,0.60,0.02}{#1}}
\newcommand{\StringTok}[1]{\textcolor[rgb]{0.31,0.60,0.02}{#1}}
\newcommand{\VariableTok}[1]{\textcolor[rgb]{0.00,0.00,0.00}{#1}}
\newcommand{\VerbatimStringTok}[1]{\textcolor[rgb]{0.31,0.60,0.02}{#1}}
\newcommand{\WarningTok}[1]{\textcolor[rgb]{0.56,0.35,0.01}{\textbf{\textit{#1}}}}
\usepackage{graphicx,grffile}
\makeatletter
\def\maxwidth{\ifdim\Gin@nat@width>\linewidth\linewidth\else\Gin@nat@width\fi}
\def\maxheight{\ifdim\Gin@nat@height>\textheight\textheight\else\Gin@nat@height\fi}
\makeatother
% Scale images if necessary, so that they will not overflow the page
% margins by default, and it is still possible to overwrite the defaults
% using explicit options in \includegraphics[width, height, ...]{}
\setkeys{Gin}{width=\maxwidth,height=\maxheight,keepaspectratio}
% Set default figure placement to htbp
\makeatletter
\def\fps@figure{htbp}
\makeatother
\setlength{\emergencystretch}{3em} % prevent overfull lines
\providecommand{\tightlist}{%
  \setlength{\itemsep}{0pt}\setlength{\parskip}{0pt}}
\setcounter{secnumdepth}{-\maxdimen} % remove section numbering

\title{BH Code Challenge}
\author{Allan Trapp II}
\date{7/27/2020}

\begin{document}
\maketitle

\hypertarget{executive-summary}{%
\subsection{Executive summary:}\label{executive-summary}}

\begin{enumerate}
\def\labelenumi{\arabic{enumi}.}
\tightlist
\item
  From which region are we most likely to find the highest quality
  wines?
\end{enumerate}

If one were to randomly pick wines and define a `high quality' wine of
having at least a quality score of 6, then the answer is region C as it
has the highest percent of bottles, 14.62\%. Region C has the highest
average of quality scores, 6.38, and is the second largest producer of
wines (1092) compared to F (1332), the largest producer. I hypothesize
that C is more of the mind, `quality over quantity,' as 86.9\% of its
wines have scores \textgreater{} 5. Only 67.0\% of region F's wines meet
this criteria, even though it is the largest producing region.

\begin{enumerate}
\def\labelenumi{\arabic{enumi}.}
\setcounter{enumi}{1}
\tightlist
\item
  Please select 10 wines for the party. What are they, and why did you
  select them?
\end{enumerate}

My objective was to ensure that diverse, high quality, red and white
wines were available. As such, I first subset the original wine
dataframe into red and white lists where wine quality was either 8 or 9.
To drive diversity, I performed agglomerative hierarchical clustering
with Ward's method on each list. Groupings were solely based on the
continuous attributes of fixed.acidity, volatile.acidity, citric.acid,
residual.sugar, chlorides, free.sulfur.dioxide, total.sulfur.dioxide,
density, pH, sulphates, and alcohol. This is the list of wines with
bottle ID:

Color ID Region White 2210 C\\
White 3315 F\\
White 2507 C\\
White 3397 B\\
White 5932 G\\
White 4774 C\\
White 4348 H Red 1203 C\\
Red 496 C\\
Red 1270 H

The list contains more whites as 75\% of the big list is white.

\begin{enumerate}
\def\labelenumi{\arabic{enumi}.}
\setcounter{enumi}{2}
\item
  Which of the characteristics most influence wine quality?
\item
  Despite his gruff exterior, it turns out that Matt Crisp, the CEO of
  Benson Hill, has a sweet tooth and that he only likes the sweetest of
  wines. He also has knowledge of flavor and thus dislikes the
  ``barnyard'' features of wines with high free sulfur dioxide. Taking
  into account Matt's preferences, and all the other features of
  quality, which wine would you select for him? Why?
\end{enumerate}

\hypertarget{data}{%
\subsection{Data}\label{data}}

\begin{verbatim}
## 
## Attaching package: 'dplyr'
\end{verbatim}

\begin{verbatim}
## The following objects are masked from 'package:stats':
## 
##     filter, lag
\end{verbatim}

\begin{verbatim}
## The following objects are masked from 'package:base':
## 
##     intersect, setdiff, setequal, union
\end{verbatim}

\begin{verbatim}
## 
## Attaching package: 'psych'
\end{verbatim}

\begin{verbatim}
## The following objects are masked from 'package:ggplot2':
## 
##     %+%, alpha
\end{verbatim}

\begin{verbatim}
## Welcome! Want to learn more? See two factoextra-related books at https://goo.gl/ve3WBa
\end{verbatim}

\hypertarget{loading-data}{%
\subsubsection{Loading data}\label{loading-data}}

Reading data into R from my personal google drive. I am treating this
like a `central' data storage place to enable reproducibility of work.
This is googlesheet version of the WineData.csv that was shared through
email.

\begin{Shaded}
\begin{Highlighting}[]
\NormalTok{df <-}\StringTok{ }\NormalTok{googlesheets4}\OperatorTok{::}\KeywordTok{read_sheet}\NormalTok{(}\StringTok{'179prhCDCppqPjAlnA26FEROVqIHUR8OWwyb2QWbOfd8'}\NormalTok{,}
                        \DataTypeTok{sheet =} \StringTok{'WineData'}\NormalTok{)}
\end{Highlighting}
\end{Shaded}

\begin{verbatim}
## Using an auto-discovered, cached token.
## To suppress this message, modify your code or options to clearly consent to the use of a cached token.
## See gargle's "Non-interactive auth" vignette for more details:
## https://gargle.r-lib.org/articles/non-interactive-auth.html
\end{verbatim}

\begin{verbatim}
## The googlesheets4 package is using a cached token for allantrapp2@gmail.com.
\end{verbatim}

\begin{verbatim}
## Auto-refreshing stale OAuth token.
\end{verbatim}

\begin{verbatim}
## Reading from "WineData"
\end{verbatim}

\begin{verbatim}
## Range "'WineData'"
\end{verbatim}

\begin{verbatim}
## New names:
## * `` -> ...1
\end{verbatim}

\begin{Shaded}
\begin{Highlighting}[]
\CommentTok{#df <- read.csv('~/Downloads/WineData.csv', header=T)}
\end{Highlighting}
\end{Shaded}

\hypertarget{exploration-of-data}{%
\subsubsection{Exploration of data}\label{exploration-of-data}}

Objective of this exercise is to create wine recommendations. In the
data set, ID is assumed to be the wine identifier, and \texttt{...1}
will be removed. The following code was my verification that the two
columns are identical.

\begin{Shaded}
\begin{Highlighting}[]
\KeywordTok{head}\NormalTok{(df)}
\end{Highlighting}
\end{Shaded}

\begin{verbatim}
## # A tibble: 6 x 16
##    ...1    ID color region fixed.acidity volatile.acidity citric.acid
##   <dbl> <dbl> <chr> <chr>          <dbl>            <dbl>       <dbl>
## 1     1     1 red   I                7.4             0.7         0   
## 2     2     2 red   E                7.8             0.88        0   
## 3     3     3 red   A                7.8             0.76        0.04
## 4     4     4 red   E               11.2             0.28        0.56
## 5     5     5 red   I                7.4             0.7         0   
## 6     6     6 red   I                7.4             0.66        0   
## # ... with 9 more variables: residual.sugar <dbl>, chlorides <dbl>,
## #   free.sulfur.dioxide <dbl>, total.sulfur.dioxide <dbl>, density <dbl>,
## #   pH <dbl>, sulphates <dbl>, alcohol <dbl>, quality <dbl>
\end{verbatim}

\begin{Shaded}
\begin{Highlighting}[]
\KeywordTok{ggplot}\NormalTok{(}\DataTypeTok{data =}\NormalTok{ df, }\KeywordTok{aes}\NormalTok{(}\DataTypeTok{x =}\NormalTok{ ...}\DecValTok{1}\NormalTok{, }\DataTypeTok{y =}\NormalTok{ ID)) }\OperatorTok{+}\StringTok{ }\KeywordTok{geom_point}\NormalTok{() }\OperatorTok{+}\StringTok{ }
\StringTok{  }\KeywordTok{geom_abline}\NormalTok{(}\DataTypeTok{intercept =} \DecValTok{0}\NormalTok{, }\DataTypeTok{slope =}\DecValTok{1}\NormalTok{, }\DataTypeTok{col=}\StringTok{'red'}\NormalTok{)}
\end{Highlighting}
\end{Shaded}

\includegraphics{bh_code_challenge_files/figure-latex/unnamed-chunk-3-1.pdf}

\begin{Shaded}
\begin{Highlighting}[]
\KeywordTok{print}\NormalTok{(}\KeywordTok{paste0}\NormalTok{(}\StringTok{'RMSE between ...1 and ID is '}\NormalTok{, }\KeywordTok{f_rmse}\NormalTok{(df}\OperatorTok{$}\NormalTok{...}\DecValTok{1}\NormalTok{, df}\OperatorTok{$}\NormalTok{ID))) }\CommentTok{# Function may be found in 'utility.R' script}
\end{Highlighting}
\end{Shaded}

\begin{verbatim}
## [1] "RMSE between ...1 and ID is 0"
\end{verbatim}

\begin{Shaded}
\begin{Highlighting}[]
\NormalTok{df_mod <-}\StringTok{ }\NormalTok{dplyr}\OperatorTok{::}\KeywordTok{select}\NormalTok{(df, }\OperatorTok{-}\StringTok{'...1'}\NormalTok{)}
\NormalTok{sdf <-}\StringTok{ }\NormalTok{dplyr}\OperatorTok{::}\KeywordTok{select}\NormalTok{(df, }\OperatorTok{-}\KeywordTok{c}\NormalTok{(}\StringTok{'...1'}\NormalTok{, }\StringTok{'ID'}\NormalTok{))}
\KeywordTok{head}\NormalTok{(sdf)}
\end{Highlighting}
\end{Shaded}

\begin{verbatim}
## # A tibble: 6 x 14
##   color region fixed.acidity volatile.acidity citric.acid residual.sugar
##   <chr> <chr>          <dbl>            <dbl>       <dbl>          <dbl>
## 1 red   I                7.4             0.7         0               1.9
## 2 red   E                7.8             0.88        0               2.6
## 3 red   A                7.8             0.76        0.04            2.3
## 4 red   E               11.2             0.28        0.56            1.9
## 5 red   I                7.4             0.7         0               1.9
## 6 red   I                7.4             0.66        0               1.8
## # ... with 8 more variables: chlorides <dbl>, free.sulfur.dioxide <dbl>,
## #   total.sulfur.dioxide <dbl>, density <dbl>, pH <dbl>, sulphates <dbl>,
## #   alcohol <dbl>, quality <dbl>
\end{verbatim}

Next step of analysis is understanding the data completeness. First, I
determine if there are duplicate rows in the data set.

\begin{Shaded}
\begin{Highlighting}[]
\NormalTok{df_summary <-}\StringTok{ }\KeywordTok{f_eda}\NormalTok{(sdf)}
\KeywordTok{print}\NormalTok{(}\KeywordTok{paste0}\NormalTok{(}\StringTok{"Number of unique wines = "}\NormalTok{, }\KeywordTok{length}\NormalTok{(}\KeywordTok{unique}\NormalTok{(df}\OperatorTok{$}\NormalTok{ID)),}
      \StringTok{' and number of data rows = '}\NormalTok{,}\KeywordTok{nrow}\NormalTok{(df)))}
\end{Highlighting}
\end{Shaded}

\begin{verbatim}
## [1] "Number of unique wines = 6497 and number of data rows = 6497"
\end{verbatim}

\begin{Shaded}
\begin{Highlighting}[]
\KeywordTok{print}\NormalTok{(}\KeywordTok{paste0}\NormalTok{(}\StringTok{"Number of duplicate rows = "}\NormalTok{, }\KeywordTok{nrow}\NormalTok{(df_summary}\OperatorTok{$}\NormalTok{duplicate_rows)))}
\end{Highlighting}
\end{Shaded}

\begin{verbatim}
## [1] "Number of duplicate rows = 1177"
\end{verbatim}

\begin{Shaded}
\begin{Highlighting}[]
\KeywordTok{print}\NormalTok{(}\KeywordTok{paste}\NormalTok{(}\StringTok{"There are"}\NormalTok{, df_summary}\OperatorTok{$}\NormalTok{duplicate_rows }\OperatorTok\StringTok{ }\KeywordTok{n_distinct}\NormalTok{(),}\StringTok{"vectors of traits associated with 2 or more wines."}\NormalTok{))}
\end{Highlighting}
\end{Shaded}

\begin{verbatim}
## [1] "There are 992 vectors of traits associated with 2 or more wines."
\end{verbatim}

There are 6,497 unique ID's and rows in the complete data set
--\textgreater{} records have not been duplicated. However, there are
wines that have the exact same quantitative/qualitative characteristics.
This raises the question: Is it possible for two wines to have the same
characteristics?

Next, I check completeness of records. If data are sparse, this will
inform whether or not certain variables will go into modeling and if
imputation should be considered. Additionally, it creates talking points
with the data owners--sometimes they forget to enter data.

\begin{Shaded}
\begin{Highlighting}[]
\NormalTok{df_summary}\OperatorTok{$}\NormalTok{missing_counts}
\end{Highlighting}
\end{Shaded}

\begin{verbatim}
##   color region fixed.acidity volatile.acidity citric.acid residual.sugar
## 1     0      0             0                0           0              0
##   chlorides free.sulfur.dioxide total.sulfur.dioxide density pH sulphates
## 1         0                   0                    0       0  0         0
##   alcohol quality
## 1       0       0
\end{verbatim}

Data are complete! No need for imputation.

Third step is to investigate any correlations and distributions of
variables in the data. It appears that there are two categorical
variables/factors of color and region. Region F has the most wines, and
there are more white than red wines in this study.

\begin{Shaded}
\begin{Highlighting}[]
\NormalTok{df_summary}\OperatorTok{$}\NormalTok{summaries_fact_cols}
\end{Highlighting}
\end{Shaded}

\begin{verbatim}
## $color
## 
##   red white 
##  1599  4898 
## 
## $region
## 
##    A    B    C    D    E    F    G    H    I 
##  882  998 1092  334  785 1332  358  369  347
\end{verbatim}

\begin{Shaded}
\begin{Highlighting}[]
\KeywordTok{with}\NormalTok{(}\DataTypeTok{data=}\NormalTok{df, }\KeywordTok{table}\NormalTok{(region, color))}
\end{Highlighting}
\end{Shaded}

\begin{verbatim}
##       color
## region  red white
##      A  206   676
##      B  195   803
##      C  333   759
##      D   82   252
##      E  346   439
##      F   95  1237
##      G  144   214
##      H   40   329
##      I  158   189
\end{verbatim}

It also looks like there are 12 quantitative variables. Arguably,
quality is an ordinal, categorical variable as it is a human assigned
value, note that the quality scores range between 3 and 9.

\begin{Shaded}
\begin{Highlighting}[]
\NormalTok{df_summary}\OperatorTok{$}\NormalTok{summaries_num_cols}
\end{Highlighting}
\end{Shaded}

\begin{verbatim}
##         fixed.acidity volatile.acidity citric.acid residual.sugar  chlorides
## Min.         3.800000         0.080000   0.0000000       0.600000 0.00900000
## 1st Qu.      6.400000         0.230000   0.2500000       1.800000 0.03800000
## Median       7.000000         0.290000   0.3100000       3.000000 0.04700000
## Mean         7.215307         0.339666   0.3186332       5.443235 0.05603386
## 3rd Qu.      7.700000         0.400000   0.3900000       8.100000 0.06500000
## Max.        15.900000         1.580000   1.6600000      65.800000 0.61100000
##         free.sulfur.dioxide total.sulfur.dioxide   density       pH sulphates
## Min.                1.00000               6.0000 0.9871100 2.720000 0.2200000
## 1st Qu.            17.00000              77.0000 0.9923400 3.110000 0.4300000
## Median             29.00000             118.0000 0.9948900 3.210000 0.5100000
## Mean               30.52532             115.7446 0.9946966 3.218501 0.5312683
## 3rd Qu.            41.00000             156.0000 0.9969900 3.320000 0.6000000
## Max.              289.00000             440.0000 1.0389800 4.010000 2.0000000
##         alcohol  quality
## Min.     8.0000 3.000000
## 1st Qu.  9.5000 5.000000
## Median  10.3000 6.000000
## Mean    10.4918 5.818378
## 3rd Qu. 11.3000 6.000000
## Max.    14.9000 9.000000
\end{verbatim}

\begin{Shaded}
\begin{Highlighting}[]
\NormalTok{colnames_num <-}\StringTok{ }\KeywordTok{names}\NormalTok{(df_summary}\OperatorTok{$}\NormalTok{summaries_num_cols)}
\KeywordTok{print}\NormalTok{(}\KeywordTok{paste0}\NormalTok{(}\StringTok{'Number of quantitative columns = '}\NormalTok{,}\KeywordTok{length}\NormalTok{(colnames_num)))}
\end{Highlighting}
\end{Shaded}

\begin{verbatim}
## [1] "Number of quantitative columns = 12"
\end{verbatim}

\begin{Shaded}
\begin{Highlighting}[]
\KeywordTok{cat}\NormalTok{(}\StringTok{'quantitative variables are:'}\NormalTok{, colnames_num)}
\end{Highlighting}
\end{Shaded}

\begin{verbatim}
## quantitative variables are: fixed.acidity volatile.acidity citric.acid residual.sugar chlorides free.sulfur.dioxide total.sulfur.dioxide density pH sulphates alcohol quality
\end{verbatim}

\begin{Shaded}
\begin{Highlighting}[]
\NormalTok{df }\OperatorTok\StringTok{ }\NormalTok{dplyr}\OperatorTok{::}\KeywordTok{group_by}\NormalTok{(region) }\OperatorTok
\StringTok{  }\NormalTok{dplyr}\OperatorTok{::}\KeywordTok{summarise}\NormalTok{(}\DataTypeTok{ave_score =} \KeywordTok{mean}\NormalTok{(quality, }\DataTypeTok{na.rm=}\OtherTok{TRUE}\NormalTok{),}
                   \DataTypeTok{med_score =} \KeywordTok{median}\NormalTok{(quality, }\DataTypeTok{na.rm=}\OtherTok{TRUE}\NormalTok{),}
                   \DataTypeTok{q1 =} \KeywordTok{quantile}\NormalTok{(quality, }\FloatTok{.25}\NormalTok{, }\DataTypeTok{na.rm=}\OtherTok{TRUE}\NormalTok{),}
                   \DataTypeTok{q3 =} \KeywordTok{quantile}\NormalTok{(quality, }\FloatTok{.75}\NormalTok{, }\DataTypeTok{na.rm=}\OtherTok{TRUE}\NormalTok{),}
                   \DataTypeTok{iqr =}\NormalTok{ q3}\OperatorTok{-}\NormalTok{q1,}
                   \DataTypeTok{stdev =} \KeywordTok{sd}\NormalTok{(quality,}\DataTypeTok{na.rm=}\OtherTok{TRUE}\NormalTok{),}
                   \DataTypeTok{n =} \KeywordTok{length}\NormalTok{(quality)}
\NormalTok{                   )}
\end{Highlighting}
\end{Shaded}

\begin{verbatim}
## `summarise()` ungrouping output (override with `.groups` argument)
\end{verbatim}

\begin{verbatim}
## # A tibble: 9 x 8
##   region ave_score med_score    q1    q3   iqr stdev     n
##   <chr>      <dbl>     <dbl> <dbl> <dbl> <dbl> <dbl> <int>
## 1 A           5.34         5     5     6     1 0.588   882
## 2 B           5.84         6     5     6     1 0.801   998
## 3 C           6.38         6     6     7     1 0.869  1092
## 4 D           5.39         5     5     6     1 0.665   334
## 5 E           5.67         6     5     6     1 0.860   785
## 6 F           5.88         6     5     6     1 0.873  1332
## 7 G           5.97         6     5     7     2 0.965   358
## 8 H           5.92         6     5     6     1 0.810   369
## 9 I           5.47         5     5     6     1 0.734   347
\end{verbatim}

\begin{Shaded}
\begin{Highlighting}[]
\KeywordTok{table}\NormalTok{(df}\OperatorTok{$}\NormalTok{quality)}
\end{Highlighting}
\end{Shaded}

\begin{verbatim}
## 
##    3    4    5    6    7    8    9 
##   30  216 2138 2836 1079  193    5
\end{verbatim}

\begin{Shaded}
\begin{Highlighting}[]
\NormalTok{psych}\OperatorTok{::}\KeywordTok{pairs.panels}\NormalTok{(dplyr}\OperatorTok{::}\KeywordTok{select}\NormalTok{(df,}\OperatorTok{-}\KeywordTok{c}\NormalTok{(}\StringTok{'color'}\NormalTok{,}\StringTok{'region'}\NormalTok{,}\StringTok{'ID'}\NormalTok{,}\StringTok{'...1'}\NormalTok{)), }
                    \DataTypeTok{method =} \StringTok{"pearson"}\NormalTok{, }\CommentTok{# correlation method}
                    \DataTypeTok{hist.col =} \StringTok{"#00AFBB"}\NormalTok{,}
                    \DataTypeTok{density =} \OtherTok{TRUE}\NormalTok{,  }\CommentTok{# show density plots}
                    \DataTypeTok{ellipses =} \OtherTok{TRUE} \CommentTok{# show correlation ellipses}
\NormalTok{)}
\end{Highlighting}
\end{Shaded}

\includegraphics{bh_code_challenge_files/figure-latex/unnamed-chunk-8-1.pdf}
For linear correlations (Pearson) to quality, alcohol content correlated
the strongest (0.44) and density was second strongest (-0.31). Residual
sugars and sulphates appear to be right-skewed distributions. Also,
there's a diversity of ranges for each variable. total.sulfur.dioxide
ranges between 0 and 440 whereas citric.acid is between 0 and 1.66.
Standardization will be considered for any clustering techniques.

\hypertarget{response-to-question-1}{%
\paragraph{Response to question 1:}\label{response-to-question-1}}

Looking at regions where wine is produced, the average quality rating of
wines comes from region C. Its mean score is 6.38.

\begin{Shaded}
\begin{Highlighting}[]
\NormalTok{pmf_region_quality <-}\StringTok{ }\KeywordTok{data.frame}\NormalTok{(}\KeywordTok{round}\NormalTok{(}\KeywordTok{with}\NormalTok{(}\DataTypeTok{data=}\NormalTok{df, }\KeywordTok{table}\NormalTok{(region, quality))}\OperatorTok{/}\KeywordTok{nrow}\NormalTok{(df),}\DecValTok{4}\NormalTok{))}
\NormalTok{pmf_region_quality <-}\StringTok{ }\NormalTok{pmf_region_quality }\OperatorTok\StringTok{ }
\StringTok{  }\NormalTok{dplyr}\OperatorTok{::}\KeywordTok{arrange}\NormalTok{(region,}\KeywordTok{desc}\NormalTok{(quality)) }\OperatorTok\StringTok{ }
\StringTok{  }\NormalTok{dplyr}\OperatorTok{::}\KeywordTok{group_by}\NormalTok{(region) }\OperatorTok
\StringTok{  }\NormalTok{dplyr}\OperatorTok{::}\KeywordTok{mutate}\NormalTok{(}\DataTypeTok{cum_prob=}\KeywordTok{cumsum}\NormalTok{(Freq)}\OperatorTok{*}\DecValTok{100}\NormalTok{)}

\KeywordTok{ggplot}\NormalTok{(}\DataTypeTok{data=}\NormalTok{pmf_region_quality, }\KeywordTok{aes}\NormalTok{(}\DataTypeTok{x=}\KeywordTok{as.numeric}\NormalTok{(quality)}\OperatorTok{+}\DecValTok{2}\NormalTok{, }\DataTypeTok{y=}\NormalTok{cum_prob, }\DataTypeTok{color=}\NormalTok{region)) }\OperatorTok{+}\StringTok{ }
\StringTok{  }\KeywordTok{geom_line}\NormalTok{() }\OperatorTok{+}
\StringTok{  }\KeywordTok{scale_x_reverse}\NormalTok{() }\OperatorTok{+}
\StringTok{  }\KeywordTok{ylab}\NormalTok{(}\StringTok{'Percent of X >= quality and Y = region'}\NormalTok{) }\OperatorTok{+}
\StringTok{  }\KeywordTok{xlab}\NormalTok{(}\StringTok{'Quality score'}\NormalTok{)}
\end{Highlighting}
\end{Shaded}

\includegraphics{bh_code_challenge_files/figure-latex/unnamed-chunk-9-1.pdf}

\begin{Shaded}
\begin{Highlighting}[]
\NormalTok{pmf_region_quality }\OperatorTok\StringTok{ }\NormalTok{dplyr}\OperatorTok{::}\KeywordTok{filter}\NormalTok{(quality}\OperatorTok{==}\DecValTok{6}\NormalTok{)}
\end{Highlighting}
\end{Shaded}

\begin{verbatim}
## # A tibble: 9 x 4
## # Groups:   region [9]
##   region quality   Freq cum_prob
##   <fct>  <fct>    <dbl>    <dbl>
## 1 A      6       0.0476     4.98
## 2 B      6       0.0737    10.1 
## 3 C      6       0.0699    14.6 
## 4 D      6       0.0157     1.88
## 5 E      6       0.0577     7.34
## 6 F      6       0.0931    13.8 
## 7 G      6       0.0239     3.93
## 8 H      6       0.0319     4.19
## 9 I      6       0.0231     2.57
\end{verbatim}

Looking at the plot above, there is a stronger likelihood that good
quality wine comes from region C as the olive color line is above all
others for high values of quality. To interpret the graphic, for the
quality value of 6 in region C (olive line), 14.62 means that 14.62\% of
all wines had a quality of 6 or higher and originated from region C. For
quality scores of 6, 7, 8, and 9, region C always has the highest
percent of wines. An interesting point, region C does not produce the
most wine, so it is not just volume driving the high joint
probabilities. In fact, 87\% of region C's wines have quality scores
larger than 5 whereas region F has only 67\%.

\begin{Shaded}
\begin{Highlighting}[]
\CommentTok{# Conditional probability calculations}
\NormalTok{df_cond <-}\StringTok{ }\NormalTok{df }\OperatorTok\StringTok{ }
\StringTok{  }\NormalTok{dplyr}\OperatorTok{::}\KeywordTok{group_by}\NormalTok{(region, quality) }\OperatorTok
\StringTok{  }\NormalTok{dplyr}\OperatorTok{::}\KeywordTok{summarize}\NormalTok{(}\DataTypeTok{counts =} \KeywordTok{length}\NormalTok{(quality)) }\OperatorTok
\StringTok{  }\NormalTok{dplyr}\OperatorTok{::}\KeywordTok{group_by}\NormalTok{(region) }\OperatorTok
\StringTok{  }\NormalTok{dplyr}\OperatorTok{::}\KeywordTok{mutate}\NormalTok{(}\DataTypeTok{reg_total =} \KeywordTok{sum}\NormalTok{(counts)) }\OperatorTok
\StringTok{  }\NormalTok{dplyr}\OperatorTok{::}\KeywordTok{ungroup}\NormalTok{(region) }\OperatorTok
\StringTok{  }\NormalTok{dplyr}\OperatorTok{::}\KeywordTok{mutate}\NormalTok{(}\DataTypeTok{p =}\NormalTok{ counts}\OperatorTok{/}\NormalTok{reg_total}\OperatorTok{*}\DecValTok{100}\NormalTok{)}
\end{Highlighting}
\end{Shaded}

\begin{verbatim}
## `summarise()` regrouping output by 'region' (override with `.groups` argument)
\end{verbatim}

\begin{Shaded}
\begin{Highlighting}[]
\NormalTok{df_cond }\OperatorTok\StringTok{ }
\StringTok{  }\NormalTok{dplyr}\OperatorTok{::}\KeywordTok{filter}\NormalTok{(quality }\OperatorTok{>}\StringTok{ }\DecValTok{5}\NormalTok{) }\OperatorTok
\StringTok{  }\NormalTok{dplyr}\OperatorTok{::}\KeywordTok{group_by}\NormalTok{(region) }\OperatorTok
\StringTok{  }\NormalTok{dplyr}\OperatorTok{::}\KeywordTok{summarize}\NormalTok{(}\DataTypeTok{p_cond =} \KeywordTok{sum}\NormalTok{(p))}
\end{Highlighting}
\end{Shaded}

\begin{verbatim}
## `summarise()` ungrouping output (override with `.groups` argument)
\end{verbatim}

\begin{verbatim}
## # A tibble: 9 x 2
##   region p_cond
##   <chr>   <dbl>
## 1 A        36.6
## 2 B        65.6
## 3 C        86.9
## 4 D        36.5
## 5 E        60.8
## 6 F        67.0
## 7 G        71.2
## 8 H        73.7
## 9 I        48.1
\end{verbatim}

\begin{Shaded}
\begin{Highlighting}[]
\KeywordTok{ggplot}\NormalTok{(}\DataTypeTok{data=}\NormalTok{df, }\KeywordTok{aes}\NormalTok{(}\DataTypeTok{x=}\NormalTok{quality)) }\OperatorTok{+}\StringTok{ }\KeywordTok{geom_bar}\NormalTok{(}\DataTypeTok{position=}\StringTok{'stack'}\NormalTok{,}\KeywordTok{aes}\NormalTok{(}\DataTypeTok{fill=}\NormalTok{region)) }
\end{Highlighting}
\end{Shaded}

\includegraphics{bh_code_challenge_files/figure-latex/unnamed-chunk-11-1.pdf}

\hypertarget{party-wine-selection-clustering-of-wines-response-to-question-2}{%
\subsection{Party wine selection: Clustering of wines, response to
question
2}\label{party-wine-selection-clustering-of-wines-response-to-question-2}}

\begin{Shaded}
\begin{Highlighting}[]
\CommentTok{# Only selecting wines with a quality score of 8 or higher}
\NormalTok{party_wines <-}\StringTok{ }\NormalTok{df }\OperatorTok\StringTok{ }\NormalTok{dplyr}\OperatorTok{::}\KeywordTok{filter}\NormalTok{(quality }\OperatorTok{>=}\DecValTok{8}\NormalTok{)}

\NormalTok{cluster_cols <-}\StringTok{ }\KeywordTok{setdiff}\NormalTok{(colnames_num, }\StringTok{'quality'}\NormalTok{)}

\CommentTok{# Normalizing quantitative variables}
\NormalTok{scale2 <-}\StringTok{ }\ControlFlowTok{function}\NormalTok{(x, }\DataTypeTok{na.rm =} \OtherTok{TRUE}\NormalTok{) (x }\OperatorTok{-}\StringTok{ }\KeywordTok{mean}\NormalTok{(x, }\DataTypeTok{na.rm =}\NormalTok{ na.rm)) }\OperatorTok{/}\StringTok{ }\KeywordTok{sd}\NormalTok{(x, na.rm)}
\NormalTok{party_wines <-}\StringTok{ }\NormalTok{party_wines }\OperatorTok\StringTok{ }
\StringTok{  }\NormalTok{dplyr}\OperatorTok{::}\KeywordTok{mutate_at}\NormalTok{(}\KeywordTok{all_of}\NormalTok{(cluster_cols), scale2)}

\CommentTok{# Creating two separate wine lists for whites and reds}
\NormalTok{party_wines_red <-}\StringTok{ }\NormalTok{party_wines }\OperatorTok\StringTok{ }\NormalTok{dplyr}\OperatorTok{::}\KeywordTok{filter}\NormalTok{(color}\OperatorTok{==}\StringTok{'red'}\NormalTok{)}
\NormalTok{party_wines_white <-}\StringTok{ }\NormalTok{party_wines }\OperatorTok\StringTok{ }\NormalTok{dplyr}\OperatorTok{::}\KeywordTok{filter}\NormalTok{(color}\OperatorTok{==}\StringTok{'white'}\NormalTok{)}

\CommentTok{# Only 18 red wines have quality >=8}
\KeywordTok{dim}\NormalTok{(party_wines_red)}
\end{Highlighting}
\end{Shaded}

\begin{verbatim}
## [1] 18 16
\end{verbatim}

\begin{Shaded}
\begin{Highlighting}[]
\CommentTok{# Only 180 white wines have quality >=8}
\KeywordTok{dim}\NormalTok{(party_wines_white)}
\end{Highlighting}
\end{Shaded}

\begin{verbatim}
## [1] 180  16
\end{verbatim}

\begin{Shaded}
\begin{Highlighting}[]
\CommentTok{# Using Agnes coefficient to determine good clustering method}
\NormalTok{m <-}\StringTok{ }\KeywordTok{c}\NormalTok{(}\StringTok{'average'}\NormalTok{, }\StringTok{'single'}\NormalTok{, }\StringTok{'complete'}\NormalTok{, }\StringTok{'ward'}\NormalTok{)}
\KeywordTok{names}\NormalTok{(m) <-}\StringTok{ }\KeywordTok{c}\NormalTok{(}\StringTok{'average'}\NormalTok{, }\StringTok{'single'}\NormalTok{, }\StringTok{'complete'}\NormalTok{, }\StringTok{'ward'}\NormalTok{)}

\CommentTok{# Agnes clustering function using red list}
\NormalTok{ac <-}\StringTok{ }\ControlFlowTok{function}\NormalTok{(x)\{}
\NormalTok{  cluster}\OperatorTok{::}\KeywordTok{agnes}\NormalTok{(party_wines_red[,cluster_cols], }\DataTypeTok{method=}\NormalTok{x)}\OperatorTok{$}\NormalTok{ac}
\NormalTok{\}}

\CommentTok{# Agnes clustering function using white list}
\NormalTok{ac1 <-}\StringTok{ }\ControlFlowTok{function}\NormalTok{(x)\{}
\NormalTok{  cluster}\OperatorTok{::}\KeywordTok{agnes}\NormalTok{(party_wines_white[,cluster_cols], }\DataTypeTok{method=}\NormalTok{x)}\OperatorTok{$}\NormalTok{ac}
\NormalTok{\}}
\NormalTok{purrr}\OperatorTok{::}\KeywordTok{map_dbl}\NormalTok{(m, ac)}
\end{Highlighting}
\end{Shaded}

\begin{verbatim}
##   average    single  complete      ward 
## 0.6209827 0.3855310 0.7795045 0.8291297
\end{verbatim}

\begin{Shaded}
\begin{Highlighting}[]
\NormalTok{purrr}\OperatorTok{::}\KeywordTok{map_dbl}\NormalTok{(m, ac1)}
\end{Highlighting}
\end{Shaded}

\begin{verbatim}
##   average    single  complete      ward 
## 0.8323981 0.8136266 0.8855133 0.9648037
\end{verbatim}

\begin{Shaded}
\begin{Highlighting}[]
\CommentTok{# Looks like Ward is best clustering algorithm for both list based on the Agnes coefficient.}
\end{Highlighting}
\end{Shaded}

\begin{Shaded}
\begin{Highlighting}[]
\NormalTok{hc_red <-}\StringTok{ }\NormalTok{cluster}\OperatorTok{::}\KeywordTok{agnes}\NormalTok{(party_wines_red[,cluster_cols], }\DataTypeTok{method =} \StringTok{'ward'}\NormalTok{)}
\NormalTok{hc_white <-}\StringTok{ }\NormalTok{cluster}\OperatorTok{::}\KeywordTok{agnes}\NormalTok{(party_wines_white[,cluster_cols], }\DataTypeTok{method =} \StringTok{'ward'}\NormalTok{)}

\CommentTok{# Plotting the tree diagrams}
\KeywordTok{pltree}\NormalTok{(hc_red, }\DataTypeTok{hang=}\OperatorTok{-}\DecValTok{1}\NormalTok{, }\DataTypeTok{main=}\StringTok{'Dendogram of red list using Ward'}\NormalTok{,}\DataTypeTok{xlab=}\StringTok{'Row ID'}\NormalTok{)}
\end{Highlighting}
\end{Shaded}

\includegraphics{bh_code_challenge_files/figure-latex/unnamed-chunk-14-1.pdf}

\begin{Shaded}
\begin{Highlighting}[]
\KeywordTok{pltree}\NormalTok{(hc_white, }\DataTypeTok{cex=}\NormalTok{.}\DecValTok{5}\NormalTok{, }\DataTypeTok{hang=}\OperatorTok{-}\DecValTok{1}\NormalTok{, }\DataTypeTok{main=}\StringTok{'Dendogram of white list using Ward'}\NormalTok{,}\DataTypeTok{xlab=}\StringTok{'Row ID'}\NormalTok{)}
\end{Highlighting}
\end{Shaded}

\includegraphics{bh_code_challenge_files/figure-latex/unnamed-chunk-14-2.pdf}

\begin{Shaded}
\begin{Highlighting}[]
\CommentTok{# Number of clusters selected based on the dendogram}
\NormalTok{party_wines_red <-}\StringTok{ }\NormalTok{party_wines_red }\OperatorTok\StringTok{ }\NormalTok{dplyr}\OperatorTok{::}\KeywordTok{mutate}\NormalTok{(}\DataTypeTok{cluster =} \KeywordTok{cutree}\NormalTok{(}\KeywordTok{as.hclust}\NormalTok{(hc_red), }\DataTypeTok{k=}\DecValTok{3}\NormalTok{)) }
\NormalTok{party_wines_white <-}\StringTok{ }\NormalTok{party_wines_white }\OperatorTok\StringTok{ }\NormalTok{dplyr}\OperatorTok{::}\KeywordTok{mutate}\NormalTok{(}\DataTypeTok{cluster =} \KeywordTok{cutree}\NormalTok{(}\KeywordTok{as.hclust}\NormalTok{(hc_white), }\DataTypeTok{k=}\DecValTok{7}\NormalTok{))}

\CommentTok{# Elbow plot to determine optimal number of clusters in the Red list}
\KeywordTok{fviz_nbclust}\NormalTok{(party_wines_red[,cluster_cols], }\DataTypeTok{FUN =}\NormalTok{ hcut, }\DataTypeTok{method =} \StringTok{"wss"}\NormalTok{) }\OperatorTok{+}
\StringTok{  }\KeywordTok{ggtitle}\NormalTok{(}\StringTok{"Elbow plot for red wine clustering"}\NormalTok{)}
\end{Highlighting}
\end{Shaded}

\includegraphics{bh_code_challenge_files/figure-latex/unnamed-chunk-14-3.pdf}

\begin{Shaded}
\begin{Highlighting}[]
\CommentTok{# Elbow plot to determine optimal number of clusters in the White list}
\KeywordTok{fviz_nbclust}\NormalTok{(party_wines_white[,cluster_cols], }\DataTypeTok{FUN =}\NormalTok{ hcut, }\DataTypeTok{method =} \StringTok{"wss"}\NormalTok{) }\OperatorTok{+}
\StringTok{  }\KeywordTok{ggtitle}\NormalTok{ (}\StringTok{"Elbow plot for white wine clustering"}\NormalTok{)}
\end{Highlighting}
\end{Shaded}

\includegraphics{bh_code_challenge_files/figure-latex/unnamed-chunk-14-4.pdf}
For the red list of wines with quality score of 8 or higher, elbow
method shows that 3 clusters seems reasonable. As such, I will restrict
the white list to 7 clusters to get 10 distinct groups of high quality
red and white wines from which we will select diverse wines.

\begin{Shaded}
\begin{Highlighting}[]
\KeywordTok{with}\NormalTok{(}\DataTypeTok{data=}\NormalTok{party_wines_red, }\KeywordTok{table}\NormalTok{(region, cluster))}
\end{Highlighting}
\end{Shaded}

\begin{verbatim}
##       cluster
## region 1 2 3
##      C 5 4 0
##      E 1 1 1
##      G 1 2 0
##      H 0 0 3
\end{verbatim}

\begin{Shaded}
\begin{Highlighting}[]
\KeywordTok{with}\NormalTok{(}\DataTypeTok{data=}\NormalTok{party_wines_white, }\KeywordTok{table}\NormalTok{(region, cluster))}
\end{Highlighting}
\end{Shaded}

\begin{verbatim}
##       cluster
## region  1  2  3  4  5  6  7
##      A  0  0  0  1  0  0  0
##      B  3  2  0 16  0  4  0
##      C 22  1 25  4  0 23  0
##      E  6  0  3  0  0  2  0
##      F  6 24  8  0  0  3  0
##      G  0  0  0  0 14  0  0
##      H  0  2  0  1  3  0  5
##      I  0  0  0  0  2  0  0
\end{verbatim}

It seems as though our clustering algorithm has some association to
where the wines originated. For the white list, cluster 7 only has
region H wines. Cluster 5 is predominantly region G and I wines. Cluster
4 is predominantly region A and B wines. Cluster 2 is predominantly
region F wines. As such, I will not give any further consideration to
using region as a factor in selecting the wine list for the party.

\begin{Shaded}
\begin{Highlighting}[]
\CommentTok{# Visualizing the clusters using first two principle components as x and y axes}
\NormalTok{factoextra}\OperatorTok{::}\KeywordTok{fviz_cluster}\NormalTok{(}\KeywordTok{list}\NormalTok{(}\DataTypeTok{data =}\NormalTok{ party_wines_red[,cluster_cols], }\DataTypeTok{cluster=}\NormalTok{party_wines_red}\OperatorTok{$}\NormalTok{cluster))}
\end{Highlighting}
\end{Shaded}

\includegraphics{bh_code_challenge_files/figure-latex/unnamed-chunk-16-1.pdf}

\begin{Shaded}
\begin{Highlighting}[]
\NormalTok{factoextra}\OperatorTok{::}\KeywordTok{fviz_cluster}\NormalTok{(}\KeywordTok{list}\NormalTok{(}\DataTypeTok{data =}\NormalTok{ party_wines_white[,cluster_cols], }\DataTypeTok{cluster=}\NormalTok{party_wines_white}\OperatorTok{$}\NormalTok{cluster))}
\end{Highlighting}
\end{Shaded}

\includegraphics{bh_code_challenge_files/figure-latex/unnamed-chunk-16-2.pdf}

\begin{Shaded}
\begin{Highlighting}[]
\CommentTok{# Calculating centroids of each cluster}
\NormalTok{centroids_red <-}\StringTok{ }\NormalTok{party_wines_red }\OperatorTok\StringTok{ }
\StringTok{  }\NormalTok{dplyr}\OperatorTok{::}\KeywordTok{select}\NormalTok{(}\KeywordTok{c}\NormalTok{(cluster_cols, }\StringTok{'cluster'}\NormalTok{, }\StringTok{'ID'}\NormalTok{)) }\OperatorTok
\StringTok{  }\NormalTok{dplyr}\OperatorTok{::}\KeywordTok{group_by}\NormalTok{(cluster) }\OperatorTok
\StringTok{  }\KeywordTok{summarise_at}\NormalTok{(}\KeywordTok{vars}\NormalTok{(}\OperatorTok{-}\NormalTok{ID),}\KeywordTok{funs}\NormalTok{(}\KeywordTok{mean}\NormalTok{(.,}\DataTypeTok{na.rm=}\OtherTok{TRUE}\NormalTok{)))}
\end{Highlighting}
\end{Shaded}

\begin{verbatim}
## Note: Using an external vector in selections is ambiguous.
## i Use `all_of(cluster_cols)` instead of `cluster_cols` to silence this message.
## i See <https://tidyselect.r-lib.org/reference/faq-external-vector.html>.
## This message is displayed once per session.
\end{verbatim}

\begin{verbatim}
## Warning: `funs()` is deprecated as of dplyr 0.8.0.
## Please use a list of either functions or lambdas: 
## 
##   # Simple named list: 
##   list(mean = mean, median = median)
## 
##   # Auto named with `tibble::lst()`: 
##   tibble::lst(mean, median)
## 
##   # Using lambdas
##   list(~ mean(., trim = .2), ~ median(., na.rm = TRUE))
## This warning is displayed once every 8 hours.
## Call `lifecycle::last_warnings()` to see where this warning was generated.
\end{verbatim}

\begin{Shaded}
\begin{Highlighting}[]
\NormalTok{centroids_white <-}\StringTok{ }\NormalTok{party_wines_white }\OperatorTok\StringTok{ }
\StringTok{  }\NormalTok{dplyr}\OperatorTok{::}\KeywordTok{select}\NormalTok{(}\KeywordTok{c}\NormalTok{(cluster_cols, }\StringTok{'cluster'}\NormalTok{, }\StringTok{'ID'}\NormalTok{)) }\OperatorTok
\StringTok{  }\NormalTok{dplyr}\OperatorTok{::}\KeywordTok{group_by}\NormalTok{(cluster) }\OperatorTok
\StringTok{  }\KeywordTok{summarise_at}\NormalTok{(}\KeywordTok{vars}\NormalTok{(}\OperatorTok{-}\NormalTok{ID),}\KeywordTok{funs}\NormalTok{(}\KeywordTok{mean}\NormalTok{(.,}\DataTypeTok{na.rm=}\OtherTok{TRUE}\NormalTok{)))}

\CommentTok{# Finding the bottle of wine closest to the centoid}
\NormalTok{r_dist <-}\StringTok{ }\KeywordTok{as.matrix}\NormalTok{(stats}\OperatorTok{::}\KeywordTok{dist}\NormalTok{(}\KeywordTok{rbind}\NormalTok{(dplyr}\OperatorTok{::}\KeywordTok{select}\NormalTok{(centroids_red,}\OperatorTok{-}\NormalTok{cluster),}
\NormalTok{                                      party_wines_red[,cluster_cols]),}
                                \DataTypeTok{method=}\StringTok{'euclidean'}\NormalTok{))[}\OperatorTok{-}\KeywordTok{c}\NormalTok{(}\DecValTok{1}\OperatorTok{:}\DecValTok{3}\NormalTok{),}\DecValTok{1}\OperatorTok{:}\DecValTok{3}\NormalTok{]}
\NormalTok{red_pos <-}\StringTok{ }\KeywordTok{apply}\NormalTok{(r_dist,}\DecValTok{2}\NormalTok{,which.min)}

\NormalTok{w_dist <-}\StringTok{ }\KeywordTok{as.matrix}\NormalTok{(stats}\OperatorTok{::}\KeywordTok{dist}\NormalTok{(}\KeywordTok{rbind}\NormalTok{(dplyr}\OperatorTok{::}\KeywordTok{select}\NormalTok{(centroids_white, }\OperatorTok{-}\NormalTok{cluster),}
\NormalTok{                                      party_wines_white[,cluster_cols]),}
                                \DataTypeTok{method=}\StringTok{'euclidean'}\NormalTok{))[}\OperatorTok{-}\KeywordTok{c}\NormalTok{(}\DecValTok{1}\OperatorTok{:}\DecValTok{7}\NormalTok{),}\DecValTok{1}\OperatorTok{:}\DecValTok{7}\NormalTok{]}
\NormalTok{white_pos <-}\StringTok{ }\KeywordTok{apply}\NormalTok{(w_dist,}\DecValTok{2}\NormalTok{,which.min)}

\CommentTok{# This is the wine list!}
\NormalTok{party_wines_red[red_pos,}\KeywordTok{c}\NormalTok{(}\StringTok{"ID"}\NormalTok{, }\StringTok{"cluster"}\NormalTok{,}\StringTok{"region"}\NormalTok{)]}
\end{Highlighting}
\end{Shaded}

\begin{verbatim}
## # A tibble: 3 x 3
##      ID cluster region
##   <dbl>   <int> <chr> 
## 1  1203       1 C     
## 2   496       2 C     
## 3  1270       3 H
\end{verbatim}

\begin{Shaded}
\begin{Highlighting}[]
\NormalTok{party_wines_white[white_pos,}\KeywordTok{c}\NormalTok{(}\StringTok{"ID"}\NormalTok{,}\StringTok{"cluster"}\NormalTok{,}\StringTok{"region"}\NormalTok{)]}
\end{Highlighting}
\end{Shaded}

\begin{verbatim}
## # A tibble: 7 x 3
##      ID cluster region
##   <dbl>   <int> <chr> 
## 1  2210       1 C     
## 2  3315       2 F     
## 3  2507       3 C     
## 4  3397       4 B     
## 5  5932       5 G     
## 6  4774       6 C     
## 7  4348       7 H
\end{verbatim}

Based the lists provided, I would recommend ID's 1203, 496, and 1270 for
the red wines and ID's 2210, 3315, 2507, 3397, 5932, 4774, and 4348 for
the white wines.

\hypertarget{attributes-of-a-good-quality-wine-response-to-question-3}{%
\subsection{Attributes of a good quality wine, response to question
3}\label{attributes-of-a-good-quality-wine-response-to-question-3}}

\end{document}
